\documentclass[a4paper,DIV10,12pt,headsepline]{scrartcl}

\usepackage[utf8]{inputenc}
\usepackage[ngerman]{babel}
\usepackage{graphicx}
\usepackage{fancyhdr}
\usepackage{textpos}

\newcommand{\nevname}{Freifunk Chemnitz}
\newcommand{\evname}{\nevname \ e.V.}
\newcommand{\datum}{07.04.2012}
\newcommand{\logo}{\includegraphics[scale=0.1]{ffc.png}}
\newcommand{\biglogo}{\begin{figure}[h] \includegraphics[width=0.8\textwidth]{ffc.png} \end{figure}}

\title{Beitragsordnung}
\author{\nevname}
\date{\today}

\begin{document}
\pagestyle{empty}
\maketitle

\pagestyle{fancy}
\lhead{\logo}
\rhead{Vereinssatzung \nevname}

\section{Allgemeines}
Die Mittel für die Verwirklichung der Zwecke des Vereins sollen durch Beiträge und sonstige Zuwendungen aufgebracht werden. Durch die Zahlung des Mitgliedsbeitrages entstehen für die Mitglieder keine Ansprüche auf Sach- oder anders geartete Leistungen. 

\section{Höhe der Mitgliedsbeiträge}
Der Beitrag für eine Ordentliche Mitgliedschaft beträgt 10 Euro pro Kalendermonat. Der Beitrag für eine Fördermitgliedschaft liegt bei 50 Euro pro Kalendermonat.

\section{Beginn und Ende der Beitragspflicht}
Die Beitragspflicht beginnt in dem Monat, der der Entscheidung des Vorstandes über die Aufnahme folgt. Sie endet in dem Monat, in dem die Mitgliedschaft beendet wird.

\section{Fälligkeit und Zahlung des Beitrages; Mahnung}
\begin{enumerate}
	\item Der Mitgliedsbeitrag wird jeweils zu Beginn eines jeden Kalendermonats im Voraus fällig.
	\item Der Mitgliedsbeitrag kann durch die Bereitstellung von Sachmitteln oder Arbeitszeit im Einvernehmen mit dem Vorstand abgegolten werden.
	\item Kommt ein Mitglied mit der Bezahlung des Mitgliedsbeitrages in Verzug, so erfolgt eine erste schriftliche Mahnung, in der ein späterer Zahlungszeitpunkt von einem Monat festgelegt wird. Erfolgt bis zum festgesetzten Zeitpunkt kein Zahlungseingang, erfolgt eine zweite schriftliche Mahnung im Folgemonat.
	\item Ein in Verzug gekommenes Mitglied, begleicht immer die ältesten Forderungen zuerst.
	
\end{enumerate}

\section{Gültigkeit der Beitragsordnung}
Die Beitragsordnung gilt ab dem Tage der Beschlussfassung durch die Mitgliederversammlung. Die Beitragsordnung hat Gültigkeit, bis durch die Mitgliederversammlung eine Änderung beschlossen wird.

\end{document}