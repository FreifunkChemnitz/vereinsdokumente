\documentclass[a4paper,DIV10,12pt,headsepline]{scrartcl}

\usepackage[utf8]{inputenc}
\usepackage[ngerman]{babel}
\usepackage{graphicx}
\usepackage{fancyhdr}
\usepackage{textpos}
\usepackage{tabularx}

\newcommand{\nevname}{Freifunk Chemnitz}
\newcommand{\evname}{\nevname \ e.V.}
\newcommand{\datum}{07.04.2012}
\newcommand{\logo}{\includegraphics[scale=0.1]{ffc.png}}
\newcommand{\biglogo}{\begin{figure}[h] \includegraphics[width=0.8\textwidth]{ffc.png} \end{figure}}

\title{Vereinssatzung}
\author{\nevname}
\date{\today}

\begin{document}
\pagestyle{empty}
\maketitle

\pagestyle{fancy}
\lhead{\logo}
\rhead{Vereinssatzung \nevname}

\begin{textblock}{4}(2.3,-5.4)
\biglogo
\end{textblock}

\section{Name und Sitz des Vereins, Geschäftsjahr}
\begin{enumerate}
	\item Der Verein führt den Namen {\glqq}\nevname{\grqq}. Er soll in das Vereinsregister eingetragen werden und danach den Zusatz {\glqq}e.V.{\grqq} führen.
	\item Der Verein hat seinen Sitz in Chemnitz.
	\item Das Geschäftsjahr ist das Kalenderjahr.
\end{enumerate}

\section{Zweck des Vereins, Gemeinnützigkeit}
\begin{enumerate}
	\item Zweck des Vereins ist die Förderung der Bildung und Kultur bezüglich kabelloser und kabelgebundener Computernetzwerke, die der Allgemeinheit zugänglich sind (freie Netzwerke) wie Absatz 8 näher erläutert.
	\item Der Verein verfolgt ausschließlich und unmittelbar gemeinnützige Zwecke im Sinne des Abschnittes {\glqq}Steuerbegünstigte Zwecke{\grqq} der Abgabenordnung. 
	\item Der Verein ist selbstlos tätig; er verfolgt nicht in erster Linie eigenwirtschaftliche Zwecke.
	\item Der Verein fördert die Wissenschaft und Forschung im Bereich kabelloser Netzwerke durch Zusammenarbeit in Projekten mit Hochschulen (z.B. HS Mittweida, WH Zwickau und TU Chemnitz). Dazu werden zum Beispiel Abschlussarbeiten mit Hilfe des Vereins betreut.
	\item Der Verein fördert den Zugang von benachteiligten Personengruppen zur Informationsgesellschaft durch die Bereitstellung eines frei zugänglichen Netzwerk und dem Wissenstransfer zur verantwortungsvollen Nutzung des Netzes.
	\item Mittel des Vereins dürfen nur für die satzungsmäßigen Zwecke verwendet werden. Die Mitglieder erhalten keine Zuwendungen aus den Mitteln des Vereins.
	\item Die Mitglieder erhalten keine Gewinnanteile. Es darf keine Person durch Ausgaben, die dem Zweck des Vereins fremd sind, oder durch unverhältnismäßig hohe Vergütungen begünstigt werden. Bei Ausscheiden oder Auflösung dürfen Vereinsmitglieder keine Anteile des Vereinsvermögens erhalten.
	\item Der Satzungszweck wird verwirklicht insbesondere durch folgende Maßnahmen:
	\begin{enumerate}
		\item Information der Mitglieder, der Öffentlichkeit und interessierter Kreise über freie Netzwerke, insbesondere durch das Internet und durch Vorträge, Veranstaltungen,
Vorführungen und Publikationen;
		\item Bereitstellung von Know-How über Technik und Anwendung freier Netzwerke (Routingprotokolle, Betriebssysteme, 802.11x, Meshverfahren, VPNs, Webservices und andere moderne Verfahren zum Betrieb einer IT-Infrastruktur);
		\item Information über gesellschaftliche, kulturelle, gesundheitliche, rechtliche und weitere Auswirkungen freier Netzwerke;
		\item Förderung der Kontakte und des Austauschs mit weiteren Personen und Organisationen In- und Ausland, die im Bereich der freien Netzwerke tätig sind (Funkfeuer Österreich, Openwireless Schweiz, Wlan Slovenja und andere Freifunk Initiativen in Deutschland).
		\item Förderung und Unterstützung von Projekten und Initiativen, die in ähnlichen Bereichen tätig sind oder denen die Idee freier Netzwerke näher gebracht werden soll (Radio T e.V., Bandbüro Chemnitz e.V., Jugendzentren und Begegnungsstätten wie das Bürgerhaus Brühl Nord).
	\end{enumerate}	
	\item Jeder Beschluss über die Änderung der Satzung ist vor dessen Anmeldung beim Registergericht dem zuständigen Finanzamt vorzulegen.
\end{enumerate}

\section{Mitgliedschaft}
\begin{enumerate}
	\item Mitglieder des Vereins sind ordentliche Mitglieder und Fördermitglieder. Ordentliche Mitglieder sind in der Mitgliederversammlung stimmberechtigt.
	\item Ordentliches Mitglied des Vereins kann jede natürliche Person werden, die sich mit den Zielen des Vereins verbunden fühlt und den Verein aktiv fördern will. Die Mitgliedschaft ist in Textform (§ 126b BGB) zu beantragen. Über den Antrag entscheidet der Vorstand. Der Antrag soll den Namen und die Anschrift des Antragstellers enthalten und angeben, wie der Antragsteller den Vereinszweck aktiv fördern will.
	\item Fördermitglied des Vereins kann jede natürliche oder juristische Person werden, die sich mit den Zielen des Vereins verbunden fühlt und den Verein finanziell und ideell unterstützen will. Die Mitgliedschaft ist in Textform (§ 126b BGB) zu beantragen. Über den Antrag entscheidet ein Vorstandsmitglied. Der Antrag soll den Namen und die Anschrift des Antragstellers enthalten.
	\item Gegen den ablehnenden Bescheid des Vorstands, der mit Gründen zu versehen ist, kann der Antragsteller Beschwerde erheben. Die Beschwerde ist innerhalb eines Monats ab Zugang des ablehnenden Bescheids schriftlich beim Vorstand einzulegen. Über die Beschwerde entscheidet die nächste ordentliche Mitgliederversammlung.
\end{enumerate}

\section{Beendigung der Mitgliedschaft}
\begin{enumerate}
	\item Die Mitgliedschaft endet:
	\begin{enumerate}
		\item mit dem Tod des Mitglieds;
		\item durch freiwilligen Austritt;
		\item durch Ausschluss aus dem Verein;
		\item bei Ausbleiben des Mitgliedsbetrags länger als 13 Monate.
	\end{enumerate}
	\item Der freiwillige Austritt erfolgt durch gegenüber einem Mitglied des Vorstands in Textform. Er ist nur zum Schluss eines Quartals unter Einhaltung einer Kündigungsfrist von zwei Wochen zulässig.
	\item Ein Mitglied kann, wenn es gegen die Vereinsinteressen gröblich verstoßen hat, durch Beschluss des Vorstands aus dem Verein ausgeschlossen werden. Vor der Beschlussfassung ist dem Mitglied unter Setzung einer angemessenen Frist Gelegenheit zu geben, sich persönlich vor dem Vorstand oder in Textform zu rechtfertigen. Eine in Textform vorliegende Stellungnahme des Betroffenen ist in der Vorstandssitzung zu verlesen. Der Beschluss über den Ausschluss ist mit Gründen zu versehen und dem Mitglied mittels eingeschriebenen Briefes bekannt zu machen. Gegen den Ausschließungsbeschluss des Vorstands steht dem Mitglied das Recht der Berufung an die Mitgliederversammlung zu. Die Berufung hat aufschiebende Wirkung. Die Berufung muss innerhalb einer Frist von einem Monat ab Zugang des Ausschließungsbeschlusses beim Vorstand schriftlich eingelegt werden. Ist die Berufung rechtzeitig eingelegt, so hat der Vorstand innerhalb von zwei Monaten die Mitgliederversammlung zur Entscheidung über die Berufung einzuberufen. Geschieht das nicht, gilt der Ausschließungsbeschluss als nicht erlassen. Macht das Mitglied von dem Recht der Berufung gegen den Ausschließungsbeschluss keinen Gebrauch oder versäumt es die Berufungsfrist, so unterwirft es sich damit dem Ausschließungsbeschluss mit der Folge, dass die Mitgliedschaft als beendet gilt.
\end{enumerate}

\section{Rechte und Pflichten der Mitglieder}
\begin{enumerate}
	\item Ordentliche Mitglieder sind berechtigt, die Leistungen des Vereins, entsprechend der vorhandenen Möglichkeiten, in angemessenen und verhältnismäßigem Ausmaß in Anspruch zu nehmen.
	\item Mitglieder sind verpflichtet, die satzungsgemäßen Zwecke des Vereins zu unterstützen und zu fördern.
	\item Der Verein erhebt einen Mitgliedsbeitrag, zu dessen Zahlung die Mitglieder verpflichtet sind. Näheres regelt eine Beitragsordnung, die von der Mitgliederversammlung beschlossen wird.
\end{enumerate}

\section{Organe des Vereins}
\begin{enumerate}
	\item Die Organe des Vereins sind:
	\begin{enumerate}
		\item Die Mitgliederversammlung,
		\item Der Vorstand.
	\end{enumerate}
\end{enumerate}

\section{Mitgliederversammlung}
\begin{enumerate}
	\item Die Mitgliederversammlung ist das oberste Beschlussorgan des Vereins. Ihr obliegen alle Entscheidungen, die nicht durch die Satzungen oder die Geschäftsordnung einem anderen Organ übertragen werden.
	\item Beschlüsse werden von der Mitgliederversammlung durch öffentliche Abstimmung getroffen. Auf Wunsch eines ordentlichen Mitglieds ist geheim abzustimmen.
	\item Jedes ordentliche Mitglied hat genau eine Stimme.
	\item Zur Fassung eines Beschlusses ist eine einfache Mehrheit der abgegebenen Stimmen notwendig. Ausgenommen sind die in \S 9 und  \S 10 geregelten Angelegenheiten. Zur Herstellung der Beschlussfähigkeit werden mindestens 40\% der Mitglieder benötigt.
	\item Eine ordentliche Mitgliederversammlung, bezeichnet als Jahreshauptversammlung, wird einmal jährlich einberufen. Ihre Tagesordnung umfasst unter anderem den Rechenschaftsbericht des Vorstands über die Vereinstätigkeit sowie den Rechenschaftsbericht des Schatzmeisters für das vorherige Geschäftsjahr.                                                 
	\item Eine außerordentliche Mitgliederversammlung kann jederzeit einberufen werden, wenn mindestens 23\% der ordentlichen Mitglieder oder der Vorstand dies jeweils gemäß \S 12 unter Angabe eines Grunds beantragen. Dem angegebenen Grund müssen die gewünschten Tagesordnungspunkte zu entnehmen sein; sie werden auf die Einladung übernommen.
	\item Dem Vorstand obliegt zu allen Mitgliederversammlungen die Festsetzung eines Termins und die rechtzeitige Einladung aller Mitglieder bis spätestens zwei Wochen vor dem von ihm festgelegten Termin. Bei von den Mitgliedern beantragten Mitgliederversammlungen darf der Termin nicht mehr als acht Wochen nach dem Eingang des Antrags beim Vorstand liegen.
	\item Der Vorstand kann die Einladungen auf schriftlichem Weg gemäß \S 12 zustellen, muss jedoch eine Kopie auf dem Postweg zustellen, falls das Mitglied den Wunsch dazu schriftlich gemäß \S 12 angemeldet hat.
	\item In der Einladung werden die Tagesordnungspunkte sowie weitere nötige Informationen bekannt gegeben. Die Mitgliederversammlung kann per Beschluss die Tagesordnung verändern.
	\item Über die Beschlüsse der Mitgliederversammlung ist ein Protokoll anzufertigen, das vom Versammlungsleiter und vom Schriftführer zu unterzeichnen ist. Das Protokoll ist innerhalb von 14 Tagen allen Mitgliedern zugänglich zu machen und auf der nächsten Mitgliederversammlung genehmigen zu lassen.
	\item Der Vorstandsvorsitzende ist Versammlungsleiter der Mitgliederversammlung. Die Mitgliederversammlung kann durch Beschluss einen anderen Versammlungsleiter oder Schriftführer bestimmen.
        \item Erreicht eine Mitgliederversammlung nicht die Beschlussfähigkeit, ist die darauffolgende ordentlich anberaumte Mitgliederversammlung beschlussfähig, wenn mindestens sieben Mitglieder anwesend sind.
\end{enumerate}

\section{Vorstand}
\begin{enumerate}
	\item Der Vorstand besteht aus mindestens drei ordentlichen Mitgliedern: dem Vorstandsvorsitzenden, dem Schatzmeister und dem Schriftführer. Des Weiteren können bis zu drei Beisitzer in den Vorstand gewählt werden. Es kann auf Wunsch der Mitgliederversammlung auf eine Wahl der Beisitzer verzichtet werden.
	\item Vorstand im Sinne des \S 26 BGB sind Vorstandsvorsitzender, Schatzmeister sowie der Schriftführer. Der Verein wird durch mindestens zwei Vorstandsmitglieder nach außen vertreten.
	\item Der Schatzmeister überwacht die Haushaltsführung und verwaltet das Vermögen des Vereins. Näheres regelt die Geschäftsordnung.
	\item Vorstandsmitglieder können jederzeit von ihrem Amt zurücktreten.
	\item Bei Rücktritt oder andauernder Ausübungsunfähigkeit von Vorstandsvorsitzendem, Schatzmeister oder Schriftführer ist der gesamte Vorstand neu zu wählen. Bis zur Wahl eines neuen Vorstands ist der bisherige Vorstand zur bestmöglichen Wahrnehmung seiner Aufgaben verpflichtet.
	\item Die Amtsdauer der Vorstandsmitglieder beträgt ein Jahr. Sie werden von der Mitgliederversammlung aus den ordentlichen Mitgliedern des Vereins gewählt. Es werden nacheinander Vorstandsvorsitzender, Schatzmeister und Schriftführer sowie falls gewünscht bis zu drei Beisitzer gewählt. Eine Wiederwahl ist beliebig oft zulässig.
	\item Der Vorstand ist Dienstvorgesetzter aller vom Verein angestellten Mitarbeiter. Er kann diese Aufgabe einem Vorstandsmitglied übertragen.
	\item Die Vorstandsmitglieder sind grundsätzlich ehrenamtlich tätig. Sie haben Anspruch auf Erstattung notwendiger Auslagen, deren Rahmen von der Geschäftsordnung festgelegt wird.
	\item Der Vorstand tritt nach Bedarf zusammen. Die Vorstandssitzungen werden vom Schriftführer schriftlich gemäß \S 12 einberufen. Der Vorstand ist beschlussfähig, wenn mindestens zwei Drittel der Vorstandsmitglieder anwesend sind. Die Beschlüsse der Vorstandssitzung sind schriftlich zu protokollieren.
	\item Jedes Vorstandsmitglied hat bei Abstimmungen des Vorstands eine Stimme. Bei Abstimmungen ist eine Mehrheit von zwei Dritteln der abgegebenen gültigen Stimmen nötig.
\end{enumerate}

\section{Änderung von Satzungs- und Geschäftsordnung}
\begin{enumerate}
	\item Über Änderungen von Satzungs- und/oder Geschäftsordnung kann in der Mitgliederversammlung nur abgestimmt werden, wenn auf diesen Tagesordnungspunkt hingewiesen wurde und der Einladung sowohl der bisherige als auch der vorgesehene neue Text beigefügt war.
	\item Für die Satzungs- oder Geschäftsordnungsänderung ist eine Mehrheit von zwei Dritteln in der Mitgliederversammlung erforderlich.
	\item Satzungsänderungen, die von Aufsichts-, Gerichts- oder Finanzbehörden aus formalen Gründen verlangt werden, kann der Vorstand von sich aus vornehmen. Diese Satzungsänderungen müssen der nächsten Mitgliederversammlung mitgeteilt werden.
\end{enumerate}

\section{Auflösung des Vereins und Vermögensbindung}
\begin{enumerate}
	\item Die Auflösung des Vereins muss von der Mitgliederversammlung mit einer Mehrheit von drei Vierteln beschlossen werden. Die Abstimmung ist nur möglich, wenn auf der Einladung zur Mitgliederversammlung als einziger Tagesordnungspunkt die Auflösung des Vereins angekündigt wurde.
	\item Bei Auflösung des Vereins, Aufhebung der Körperschaft oder Wegfall der gemeinnützigen Zwecke darf das Vermögen der Körperschaft nur für steuerbegünstigte Zwecke verwendet werden. Zur Erfüllung dieser Voraussetzung wird das Vermögen einer anderen steuerbegünstigten Körperschaft oder einer Körperschaft öffentlichen Rechts für steuerbegünstigte Zwecke übertragen, die ebenfalls den Auftrag zur Bildung und Volksbildung im Umgang mit Informationstechnologie wahrnimmt. Näheres kann die Geschäftsordnung regeln.
	\item Der Grundsatz der Vermögensbindung ist bei der Fassung von Beschlüssen über die künftige Verwendung des Vereinsvermögens zwingend zu erfüllen.
	\item Bei Verlust der Anerkennung als gemeinnütziger Verein gelten die vorgenannten Absätze analog. Das Vermögen und die Güter des Vereins werden entsprechend übertragen.
	\item Bei der Auflösung oder Aufhebung der Körperschaft fällt das Vermögen der Körperschaft an den "Förderverein Freie Netzwerke e.V." der es unmittelbar und ausschließlich zu gemeinnützigen Zwecken zu verwenden hat.
\end{enumerate}

\section{Schriftform}
\begin{enumerate}
	\item Schriftliche Erklärungen im Sinne dieser Satzung können auch elektronische Dokumente sein. Die Geschäftsordnung bestimmt Anforderungen, Zustellwege und Zuordnung derartiger Dokumente.
\end{enumerate}
%\pagebreak
%
%\section{Unterzeichner}
%Diese Satzung wurde beschlossen und gebilligt von folgenden Gründungsmitgliedern
%
%\begin{tabularx}{\textwidth}{|l|X|}
% \hline
%  Name & Unterschrift\\  \hline
%  Amadeus Alfa & \\ \hline
%  Steffen Förster & \\ \hline
%  Daniel Tändler & \\ \hline
%%  Nikita Kretschmar & \\ \hline
%  Stefan Helmert & \\ \hline
%  Mike Stummvoll & \\ \hline
%  Florian Schlegel & \\ \hline
%  Matthias Fritzsche & \\   \hline
%  Mario Höllein	 & \\   \hline
%  Maximilian Lau & \\ \hline
%  
%  
% \end{tabularx}
\end{document}
